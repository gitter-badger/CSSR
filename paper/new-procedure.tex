\documentclass{article}
%\documentclass[aos,preprint]{imsart}
\usepackage{amsmath,amsfonts,amssymb,graphics,amscd}
\usepackage[square]{natbib}
\usepackage{hyperref}
\usepackage{subfiles}
\usepackage{fullpage}

\newcommand{\Prob}[1]{\ensuremath{\mathrm{Pr}\left( #1 \right) }}
\newcommand{\Expect}[1]{\ensuremath{\mathbb{E}\left[ #1 \right]}}
\newcommand{\ProcAlphabet}{\ensuremath{\mathcal{A}}}
\newcommand{\StateSet}{\ensuremath{\mathcal{S}}}
\newcommand{\indep}{\ensuremath{\rotatebox{90}{$\models$}}}
\newcommand{\synctime}{\ensuremath{\tau_{\mathrm{synch}}}}
\newcommand{\match}{\ensuremath{\mathrm{match}}}
\newcommand{\homog}{\ensuremath{\mathrm{homog}}}
\newcommand{\excisable}{\ensuremath{\mathrm{excisable}}}
\newtheorem{proposition}{Proposition}
\newtheorem{definition}{Definition}
\newtheorem{lemma}{Lemma}
\begin{document}
\title{ROCS: A Discovery Algorithm for Predictive States}
\author{Kristina Lisa Klinkner, Alessandro Rinaldo,\\
  Cosma Rohilla Shalizi, Sam Stites\\
{\small alphabetical order, notes by CRS, revised by SS}}
\date{Notes begun 6 April 2009, this draft begun 14 May 2010, last \LaTeX 'd \today}
\maketitle
\begin{abstract}
  ROCS stands for ``Reconstruction of Causal States''; it is a successor to the
  CSSR (causal state splitting reconstruction) algorithm.  Otherwise, the
  abstract is to come.
\end{abstract}

\tableofcontents

\subfile{sections/introduction}

\subfile{sections/background}

\subfile{sections/phase-i}

\subfile{sections/phase-ii}

\subfile{sections/phase-iii}

\subfile{sections/consistency}

\bibliographystyle{acmtrans-ims}
\bibliography{locusts}

\appendix

\subfile{sections/target}


\end{document}
