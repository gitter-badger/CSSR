\documentclass[../new-procedure.tex]{subfiles}

\begin{document}

\section{Phase II: Growing the Looping Tree}

The Holmes and Isbell procedure needs some explaining before it can be adapted
to our situation.

Abstractly, their procedure relies on a notion of words being ``functionally
equivalent'', i.e., leading to the same prediction.  (In their case, this is
prediction of some output sequence from the input, whereas for us this is
prediction of the future distribution of the input, but abstractly it doesn't
matter.)  Derived notions are {\em matching}, {\em homogeneity} and {\em
  excisability}.
\begin{itemize}
\item Two words $u$, $v$ {\bf match} iff they lead to the same prediction for
  the next step.  Write this $\match(u,v)$.
\item A word $w$ is {\bf homogeneous} iff, for any prefix $u$, $\match(w,uw)$.
  Write this $\homog(w)$.  This means earlier history is irrelevant,
  conditional on $w$.
\item If $w=eq$, then the prefix $e$ is {\bf excisable} from $w$ iff, for all
  prefixes $p$, $\match(peq, pq)$.  That is, inserting the extra history $e$
  before $q$ makes no difference to predictions.
\end{itemize}

Suppose we have an oracle\footnote{This is using ``oracle'' more in the
  computer science sense than the statistical sense in e.g. ``oracle
  inequality'', so we should try to find a different name.} which can answer
matching, homogeneity and excisability queries.  The abstracted Holmes-Isbell
procedure then goes as follows.
\begin{enumerate}
\item Begin with a single node which matches the null prefix, i.e., everything.
  Set this to be a terminal node.
\item For each terminal node, check whether $w$ is homogeneous
  \begin{enumerate}
  \item If $\homog(w)$, leave $w$ alone.
  \item If $\neg\homog(w)$, delete $w$ from the set of terminal nodes (but
    leave its node in the tree).  For each symbol $a$ from the alphabet, add
    child nodes for each $aw$, unless that string does not occur in the input
    data.  For each created child node $aw$, check whether $aw=eq$ for some
    excisable prefix $e$ and suffix $q$.
    \begin{enumerate}
    \item If so, delete the node $aw$, and create a loop from $w$ to $q$ with
      the label $a$.
    \item If not (i.e., if $aw$ has no excisable prefix), check whether
      $\match(aw,r)$ for some existing terminal node $r$.
      \begin{enumerate}
      \item If so, delete the node $aw$ and create an edge labeled $a$ from $w$
        to $r$.
      \item If not, add $aw$ to the set of terminal nodes.
      \end{enumerate}
    \end{enumerate}
  \end{enumerate}
\item Continue until all terminal nodes are homogeneous.
\end{enumerate}

Note that in this looping tree, each step in the tree away from the root
corresponds to going an additional time-step back into the past.  Thus to
follow the word ``10'', we would start at the root (labeled with the empty
string), then follow the descending edge labeled ``0'', and then the descending
edge labeled ``1'' from {\em that} node.

We need to show that this procedure terminates well.  Here however I think we
can just steal from Holmes and Isbell.  I will copy this material later.  Their
key observation is that if the number of states is finite, eventually any
history extended into the past either becomes homogeneous or contains an
excisable element.  This is because, with a finite number of states, there are
only a finite number of {\em distinct} maps $M_w$.  This means that growth of a
looping tree cannot continue forever.  Moreover, it terminates with a looping
tree where each terminal node corresponds to a set of histories which have the
same distribution for the next observation, so we're next-step sufficient.

The oracle must be implemented by means of checking the conditional
probabilities.  It does not need to get the conditional probabilities exactly
right, it just needs to get queries about whether histories match right.


\subsection{Examples}

[[To be filled in, I have stepped through this for some simple machines]]

\subsection{Agenda}

\begin{enumerate}
\item Fill in the details of the proof that the algorithm halts; these should
  go exactly parallel to what Holmes and Isbell do.
\item Implement this.
\end{enumerate}

\end{document}
